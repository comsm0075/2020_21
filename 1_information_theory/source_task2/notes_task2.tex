\documentclass[12pt]{article}
\usepackage{amsfonts, epsfig}
\usepackage[authoryear]{natbib}
\usepackage{graphicx}
\usepackage{fancyhdr}
\pagestyle{fancy}
\lfoot{\texttt{comsm0075.github.io}}
\lhead{IP\&B - 1\_information\_theory - Conor}
\rhead{\thepage}
\cfoot{}
\begin{document}

\section*{Week two task - TA notes} 

\subsection*{$I(X,Y)\ge 0$}

If we have a joint distribution $p(x,y)$ for two random variables $X$
and $Y$, we have two distributions on the space of $(x,y)$ pairs, the
original distribution $p(x,y)$ and the distribution given by
marginalizing the distribution: $p(x)q(y)$.

So for convenience
of notation, that is not calling too many things `$p$', consider
$d(r\|s)$ where $r(x,y)=p(x,y)$ and $s(x,y)=p(x)p(y)$. Now
\begin{equation}
 d(r\|s) = \sum_{i,j} r(x_i,y_j)\log_2{\frac{r(x_i,y_j)}{s(x_i,y_j)}}=\sum_{i,j} p(x_i,y_j)\log_2{\frac{p(x_i,y_j)}{p(x_i)p(y_j)}}
 \end{equation}
and hence $I(X,Y)=d(r\|s)\ge 0$ with equality if and only if
$r(x,y)=s(x,y)$, that is, if the distributions are independent.

\subsection*{$H(X)\le \log_2{n}$}

So if $q(x)=u(x)$ the uniform distribution with $u(x_i)=1/n$ for all $i$ we have
\begin{equation}
  d(p|u)=\sum_i p(x_i)\log_2{p(x_i)}-\sum_i p(x_i)\log_2 \frac{1}{n}
\end{equation}
and the first term is $-H(X)$ and the second term is $-\log_2{n}$ because the $\sum_i p(x_i)=1$.

\subsection*{The coding question}

\begin{center}
\begin{tabular}{c|cccc}
&{A}&{B}&{C}&{D}\\
\hline
$q$&1/2&1/4&1/8&1/8\\
$p$&1/4&1/4&1/4&1/4\\
\hline
$q$-code&{0}&{10}&{110}&{111}\\
$p$-code&{00}&{01}&{10}&{11}
\end{tabular}
\end{center}
 Check the relationship between the divergence and the
difference in code lenghts, both using the code optimized to
$p${} and $q${}.  

So if $p(x)$ is the distribution, but we use the $q$-code, then
\begin{equation}
  L=\frac{1}{4}\left(1+2+3+3\right)=\frac{9}{4}
\end{equation}
whereas the code for $p(x)$ gives $L=2$, so the penalty for using the wrong code is $1/4$ when
\begin{equation}
  d(p\|q)=\frac{1}{4}\left(\log_2{\frac{1}{2}}+\log_2{1}+2\log_2{2}\right)=\frac{1}{4}(-1+0+2)=\frac{1}{4}
\end{equation}
Conversely is $q(x)$ is the distribution, using the $p$-code gives $L=2$ whereas the efficient code gives $L=7/4$ as before, giving, again, a gap of $1/4$.
\begin{equation}
  d(q\|p)=\frac{1}{2}\log_2{2}+\frac{1}{4}\log_2{1}+\frac{1}{4}\log_2{\frac{1}{2}}=\frac{1}{4}
\end{equation}
The symmetry seen in this problem is not a general property.

\end{document}

