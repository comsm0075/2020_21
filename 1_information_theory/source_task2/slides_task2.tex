
\documentclass{beamer}
\usepackage[latin1]{inputenc}
%\usetheme{Montpellier}
%\usetheme{Boadilla}
%\usecolortheme[RGB={204,51,255}]{structure}
%\usecolortheme[named=purple]{structure}
\usecolortheme[RGB={62,128,62}]{structure}
%\definecolor{reddish}{rgb}{0.3,0.15,0.3}
%\definecolor{light}{rgb}{0.8,0.6,0.8}
%\definecolor{reddish}{rgb}{.5,0.15,0.15}
\definecolor{reddish}{rgb}{0.5,0.3,0.4}
%\definecolor{light}{rgb}{0.8,0.6,0.8}
\definecolor{reddish}{rgb}{.7,0.25,0.25}
\definecolor{greenish}{rgb}{.25,0.7,0.25}
\definecolor{blueish}{rgb}{.25,0.25,0.7}
\definecolor{purple}{rgb}{.5,0.0,0.5}
\usepackage{graphicx}
\usepackage{pstricks}

\newcommand{\btVFill}{\vskip0pt plus 1filll}

\setbeamertemplate{navigation symbols}{}

\newcommand{\crish}{\color{reddish}}
\newcommand{\cbla}{\color{black}}
\newcommand{\cred}{\color{red}}
\newcommand{\cblu}{\color{blue}}
\newcommand{\cgre}{\color{green}}

\newcommand{\sm}{\color{reddish}$}
\newcommand{\fm}{$\color{black}{}}

\newcommand{\letter}[1]{\color{blue}\texttt{#1}\color{black}}
\newcommand{\binary}[1]{\color{red}\texttt{#1}\color{black}}

\usepackage{tikz}
\usetikzlibrary{arrows,decorations.markings,positioning}
\usepackage{epstopdf}
\usetikzlibrary{fit}

\title[Information Theory task 2]{The Kullback Leibler divergence 6}
\author{COMSM0075 Information Processing and Brain}
\institute{\texttt{comsm0075.github.io}}
\date{October 2020}

\begin{document}

\maketitle

\begin{frame}{The KL divergence}
  The \textbf{Kullback Leibler (KL) divergence} differs from the other
  information theory quantities in that it deals with two probability
  distributions $p(x)$ and $q(x)$ on the same set of outcomes
  \crish$\mathcal{X}=\{x_1,x_2,\ldots,x_n\}$\cbla.
\end{frame}

\begin{frame}{The KL divergence}
  The \textbf{Kullback Leibler (KL) divergence}, also called the \textbf{relative entropy} is
  \crish
  $$
  d(p\|q)=\sum_i p(x_i)\log_2{\frac{p(x_i)}{q(x_i)}}
    $$
    \cbla
\end{frame}

\begin{frame}{The KL divergence}
  The \textbf{Kullback Leibler (KL) divergence}, also called the \textbf{relative entropy} is
  \crish
  $$
  d(p\|q)=E_p(\log_2{p(X)})-E_p(\log{q(X)})
    $$
    \cbla
\end{frame}


\begin{frame}{The KL divergence - coding example}
\begin{quote}
  The \textbf{Kullback Leibler (KL) divergence} is the expected value
  of the number of extra bits required to encode data with
  distribution $p(x)$ compared to $q(x)$ if the code is the optimal code for
  $q(x)$.
  \end{quote}
\end{frame}

\begin{frame}{The KL divergence - coding example}

\begin{center}
\begin{tabular}{c|cccc}
&\letter{A}&\letter{B}&\letter{C}&\letter{D}\\
\hline
$q$&1/2&1/4&1/8&1/8\\
$p$&1/4&1/8&1/2&1/8\\
\hline
&\binary{0}&\binary{10}&\binary{110}&\binary{111}
\end{tabular}
\end{center}
\crish
$$
L(q)=1.75
$$
\cbla
\end{frame}


\begin{frame}{The KL divergence - coding example}
\begin{center}
\begin{tabular}{c|cccc}
&\letter{A}&\letter{B}&\letter{C}&\letter{D}\\
\hline
$q$&1/2&1/4&1/8&1/8\\
$p$&1/4&1/8&1/2&1/8\\
\hline
&\binary{0}&\binary{10}&\binary{110}&\binary{111}
\end{tabular}
\end{center}
\crish
$$
L(p)=\frac{1}{4}+\frac{1}{8}\times 2 +\frac{1}{2}\times 3+\frac{1/8}\times 3=2.375
$$
\cbla
\end{frame}


\begin{frame}{The KL divergence - coding example}
\begin{center}
\begin{tabular}{c|cccc}
&\letter{A}&\letter{B}&\letter{C}&\letter{D}\\
\hline
$q$&1/2&1/4&1/8&1/8\\
$p$&1/4&1/8&1/2&1/8\\
\hline
&\binary{0}&\binary{10}&\binary{110}&\binary{111}
\end{tabular}
\end{center}
\crish
$$
L(p)-L(q)=2.375-1.75=0.625
$$
\cbla
\end{frame}


\begin{frame}{The KL divergence - coding example}
\begin{center}
\begin{tabular}{c|cccc}
&\letter{A}&\letter{B}&\letter{C}&\letter{D}\\
\hline
$q$&1/2&1/4&1/8&1/8\\
$p$&1/4&1/8&1/2&1/8\\
\hline
&\binary{0}&\binary{10}&\binary{110}&\binary{111}
\end{tabular}
\end{center}
\crish
$$
d(p||q)=\frac{1}{4}\log_2{\frac{1}{2}}+\frac{1}{8}\log_2{\frac{1}{2}}+\frac{1}{2}\log_2{4}+\frac{1}{8}\log_2{1}
$$
\cbla
\end{frame}


\begin{frame}{The KL divergence - coding example}
\begin{center}
\begin{tabular}{c|cccc}
&\letter{A}&\letter{B}&\letter{C}&\letter{D}\\
\hline
$q$&1/2&1/4&1/8&1/8\\
$p$&1/4&1/8&1/2&1/8\\
\hline
&\binary{0}&\binary{10}&\binary{110}&\binary{111}
\end{tabular}
\end{center}
\crish
$$
d(p\|q)=1-\frac{3}{8}=0.625
$$
\cbla
\end{frame}


\begin{frame}{The information inequality}
  The \textbf{information inequality}, also called the \textbf{Gibbs inequality} says
  \crish
  $$d(p\|q)>0$$
  \cbla
  with equality if and only if \crish$p(x)=q(x)$\cbla{} for all \crish$x$\cbla{}.
\end{frame}


\begin{frame}{The information inequality}
  The \textbf{information inequality}, also called the \textbf{Gibbs inequality} says
  \crish
  $$d(p\|q)>0$$
  \cbla
  with equality if and only if \crish$p(x)=q(x)$\cbla{} for all \crish$x$\cbla{}. Follows from Jensen's inequality.
\end{frame}

\begin{frame}{Two tasks}
  \begin{enumerate}
    \item Use the information processing inequality show
      \crish$I(X,Y)\ge 0$\cbla; to do this relate
      \crish$I(X,Y)$\cbla{} to \crish$d(p\|q)$\cbla{} by treating
      \crish$p(x,y)$\cbla{} and \crish$p(x)p(y)$\cbla{} as two
      distributions on the same set of outcomes.
    \item Use the information processing inequality show
      \crish$H(X)\le \log_2{n}$\cbla{} where
      \crish$n=|\mathcal{X}|$\cbla. To do this use the uniform
      distribution as $q(x)$.
  \end{enumerate}
\end{frame}


\begin{frame}{Another coding example}

\begin{center}
\begin{tabular}{c|cccc}
&\letter{A}&\letter{B}&\letter{C}&\letter{D}\\
\hline
$q$&1/2&1/4&1/8&1/8\\
$p$&1/4&1/4&1/4&1/4\\
\hline
$q$-code&\binary{0}&\binary{10}&\binary{110}&\binary{111}\\
$p$-code&\binary{00}&\binary{01}&\binary{10}&\binary{11}
\end{tabular}
\end{center}
\cbla Check the relationship between the divergence and the
difference in code lenghts, both using the code optimized to
\crish$p$\cbla{} and \crish$q$\cbla{}.  \cbla
\end{frame}

\end{document}

