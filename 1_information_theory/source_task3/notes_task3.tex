\documentclass[12pt]{article}
\usepackage{amsfonts, epsfig}
\usepackage[authoryear]{natbib}
\usepackage{graphicx}
\usepackage{fancyhdr}
\pagestyle{fancy}
\lfoot{\texttt{comsm0075.github.io}}
\lhead{IP\&B - 1\_information\_theory - Conor}
\rhead{\thepage}
\cfoot{}
\begin{document}

\section*{Week three task - the Kelly Criterion} 

\subsection*{The question}
 The \textbf{Kelly criterion} is an investment strategy which can be
 derived in its simplest form using the methods we have discussed. In
 a simple example you have the opportunity to bet on event, that two
 dice roll a seven for example. If your bet is a success you get
 $r${} times your stake, if it fails you get nothing. The
 probability of success is $p$. What fraction of your float
 should you bet each time?

 \newpage
 
\subsection*{The answer}
So there are two values for the reward, if the bet is successful the reward is
\begin{equation}
  S_+=(1-f)+fr
\end{equation}
times the stake, if it fails $S$ is given by
\begin{equation}
  S_-=(1-f)
\end{equation}
where $f$ is the fraction of the float that is bet. Hence, the doubling rate is
\begin{equation}
  R=p\log_2{[(1-f)+fr]}+(1-p)\log_2(1-f)
\end{equation}
To minimize this we differentiate
\begin{equation}
  \frac{dR}{df}=\frac{p(r-1)}{1+(r-1)f}-\frac{1-p}{1-f}
\end{equation}
and setting this equal to zero, and for convenient $\rho=r-1$:
\begin{equation}
  (1-p)(1+\rho f)=(1-f)p\rho
\end{equation}
or
\begin{equation}
  f=p-\frac{1-p}{\rho}
\end{equation}
This is the Kelly criterion.

For $r=1/p$ the bet is fair and so over time you won't
profit. Substituting $r=1/p$ in the Kelly criterion tells you you
should bet nothing: $f=0$. Only if $r>1/p$ is $f>0$ and in that case
you should bet.
\end{document}

