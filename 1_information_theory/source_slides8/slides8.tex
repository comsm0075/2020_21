
\documentclass{beamer}
\usepackage[latin1]{inputenc}
%\usetheme{Montpellier}
%\usetheme{Boadilla}
%\usecolortheme[RGB={204,51,255}]{structure}
%\usecolortheme[named=purple]{structure}
\usecolortheme[RGB={62,128,62}]{structure}
%\definecolor{reddish}{rgb}{0.3,0.15,0.3}
%\definecolor{light}{rgb}{0.8,0.6,0.8}
%\definecolor{reddish}{rgb}{.5,0.15,0.15}
\definecolor{reddish}{rgb}{0.5,0.3,0.4}
%\definecolor{light}{rgb}{0.8,0.6,0.8}
\definecolor{reddish}{rgb}{.7,0.25,0.25}
\definecolor{greenish}{rgb}{.25,0.8,0.25}
\definecolor{blueish}{rgb}{.25,0.25,0.7}
\definecolor{purple}{rgb}{.5,0.0,0.5}
\usepackage{graphicx}
\usepackage{pstricks}

\newcommand{\btVFill}{\vskip0pt plus 1filll}

\setbeamertemplate{navigation symbols}{}

\newcommand{\crish}{\color{reddish}}
\newcommand{\cgish}{\color{greenish}}
\newcommand{\cbla}{\color{black}}
\newcommand{\cred}{\color{red}}
\newcommand{\cblu}{\color{blue}}
\newcommand{\cgrish}{\color{green}}

\newcommand{\sm}{\color{reddish}$}
\newcommand{\fm}{$\color{black}{}}

\newcommand{\letter}[1]{\color{blue}\texttt{#1}\color{black}}
\newcommand{\binary}[1]{\color{red}\texttt{#1}\color{black}}

\usepackage{tikz}
\usetikzlibrary{arrows,decorations.markings,positioning}
\usepackage{epstopdf}
\usetikzlibrary{fit}
\usepackage{pgfplots}

\title[Information Theory lecture 8]{Differential entropy and Shannon's entropy: information theory lecture 8}
\author{COMSM0075 Information Processing and Brain}
\institute{\texttt{comsm0075.github.io}}
\date{October 2020}

\begin{document}

\maketitle


\begin{frame}{Differential entropy}
\crish
$$
  h(X)=-\int dx p_X(x)\log_2{p_X(x)}
  $$ \cbla
\end{frame}


\pgfmathdeclarefunction{gauss}{2}{%
  \pgfmathparse{1/(#2*sqrt(2*pi))*exp(-((x-#1)^2)/(2*#2^2))}%
}


\pgfmathdeclarefunction{uniform}{1}{%
  \pgfmathparse{#1}%
}


\begin{frame}{A continuous distribution}
\begin{tikzpicture}
\begin{axis}[
  no markers, domain=0:10, samples=100,
  axis lines*=left, xlabel=$x$, ylabel=$y$,
  every axis y label/.style={at=(current axis.above origin),anchor=south},
  every axis x label/.style={at=(current axis.right of origin),anchor=west},
  height=5cm, width=12cm,
%  xtick={4,6.5}, ytick=\empty,
  xtick=\empty,
  ytick=\empty,
  enlargelimits=false, clip=false, axis on top,
  grid = major
  ]
%  \addplot [fill=cyan!20, draw=none, domain=0:5.96] {gauss(6.5,1)} \closedcycle;
  \addplot [very thick,cyan!50!black] {gauss(4,1)};
%  \addplot [very thick,cyan!50!black] {gauss(6.5,1)};
%\draw [yshift=-0.6cm, latex-latex](axis cs:4,0) -- node [fill=white] {$1.96\sigma$} (axis cs:5.96,0);
\end{axis}
  \draw (6,2.2) node{\color{cyan!50!black}$p_X(x)$\cbla};
\end{tikzpicture}
\end{frame}


\begin{frame}{A continuous distribution}
\begin{tikzpicture}
\begin{axis}[
  no markers, domain=0:10, samples=100,
  axis lines*=left, xlabel=$x$, ylabel=$y$,
  every axis y label/.style={at=(current axis.above origin),anchor=south},
  every axis x label/.style={at=(current axis.right of origin),anchor=west},
  height=5cm, width=12cm,
%  xtick={4,6.5}, ytick=\empty,
  xtick=\empty,
  ytick=\empty,
  enlargelimits=false, clip=false, axis on top,
  grid = major
  ]
  \foreach \x in {0,0.5,...,8}
  {
  \addplot [fill=cyan!0, draw=black, domain=\x:\x+1] {gauss(4,1)} \closedcycle;
  }
%  \addplot [fill=cyan!20, draw=black, domain=3:3.5] {gauss(4,1)} \closedcycle;
  \addplot [very thick,cyan!50!black] {gauss(4,1)};
%  \addplot [very thick,cyan!50!black] {gauss(6.5,1)};
%\draw [yshift=-0.6cm, latex-latex](axis cs:4,0) -- node [fill=white] {$1.96\sigma$} (axis cs:5.96,0);
\end{axis}

\end{tikzpicture}
\end{frame}


\begin{frame}{Discrete from continuous}
\begin{tikzpicture}
\begin{axis}[
  no markers, domain=0:10, samples=100,
  axis lines*=left, xlabel=$x$, ylabel=$y$,
  every axis y label/.style={at=(current axis.above origin),anchor=south},
  every axis x label/.style={at=(current axis.right of origin),anchor=west},
  height=5cm, width=12cm,
%  xtick={4,6.5}, ytick=\empty,
  xtick=\empty,
  ytick=\empty,
  enlargelimits=false, clip=false, axis on top,
  grid = major
  ]
  \foreach \x in {0,0.5,...,8}
  {
  \addplot [fill=cyan!0, draw=black, domain=\x:\x+1] {gauss(4,1)} \closedcycle;
  }
  \addplot [fill=cyan!20, draw=black, domain=3:3.5] {gauss(4,1)} \closedcycle;
  \addplot [very thick,cyan!50!black] {gauss(4,1)};

  %  \addplot [very thick,cyan!50!black] {gauss(6.5,1)};
%\draw [yshift=-0.6cm, latex-latex](axis cs:4,0) -- node [fill=white] {$1.96\sigma$} (axis cs:5.96,0);
\end{axis}
  \draw (3.1,-0.3) node{$x_i$};
  \draw (3.8,-0.31) node{$x_{i+1}$};
\end{tikzpicture}
\end{frame}


\begin{frame}{Discrete from continuous}
\begin{tikzpicture}
\begin{axis}[
  no markers, domain=0:10, samples=100,
  axis lines*=left, xlabel=$x$, ylabel=$y$,
  every axis y label/.style={at=(current axis.above origin),anchor=south},
  every axis x label/.style={at=(current axis.right of origin),anchor=west},
  height=5cm, width=12cm,
%  xtick={4,6.5}, ytick=\empty,
  xtick=\empty,
  ytick=\empty,
  enlargelimits=false, clip=false, axis on top,
  grid = major
  ]
  \foreach \x in {0,0.5,...,8}
  {
  \addplot [fill=cyan!0, draw=black, domain=\x:\x+1] {gauss(4,1)} \closedcycle;
  }
  \addplot [fill=cyan!20, draw=black, domain=3:3.5] {gauss(4,1)} \closedcycle;
  \addplot [very thick,cyan!50!black] {gauss(4,1)};

  %  \addplot [very thick,cyan!50!black] {gauss(6.5,1)};
%\draw [yshift=-0.6cm, latex-latex](axis cs:4,0) -- node [fill=white] {$1.96\sigma$} (axis cs:5.96,0);
\end{axis}
  \draw (3.1,-0.3) node{$x_i$};
  \draw (3.8,-0.31) node{$x_{i+1}$};
  \draw (1.7,2.6) node{\color{cyan}$q_i=\int_{x_i}^{x_{i+1}}dx\,p_x(x)$\cbla};
\end{tikzpicture}
\end{frame}

\begin{frame}{A discrete random variable}
  Let \crish$\delta=x_{i+1}-x_i$\cbla.
  \vskip 1cm Consider the discrete random variable
  \crish$X^\delta$\cbla{} whose outcomes are the
  \crish$\{x_1,x_2,\ldots\}$\cbla{} and whose probabilities are given
  by
  \crish $$
  p_{X^\delta}(x_i)=q_i
  $$\cbla
  \end{frame}


\begin{frame}{Shannon's entropy}
  \crish $$
  p_{X^\delta}(x_i)=q_i
  $$\cbla
  so
   \crish
  $$H(X^\delta)=-\sum_i q_i \log_2{q_i}$$
  \cbla
  \end{frame}

\begin{frame}{Get rid of the integrals}
  \crish $$
  q_i = \int_{x_i}^{x_{i+1}}dx\,p_x(x)
  $$\cbla
  The plot obviously is to get rid of the integral here, morally:
  \crish $$
    q_i=\int_{x_{i}}^{x_{i+1}}dx\,p_X(x)\approx p_X(\bar{x}_i)\delta 
$$\cbla
where $\bar{x}_i$ is any point in $[x_i,x_{i+1})$ and the
  approximation gets better as \crish$\delta$\cbla{} gets smaller.
  \end{frame}

\begin{frame}{Mean Value Theorem}
 \crish $$
    q_i=\int_{x_{i}}^{x_{i+1}}dx\,p_X(x)\approx p_X(\bar{x}_i)\delta 
    $$\cbla
Rather than worry about how good the approximation is, we use
    an elegant alternative approach based on the mean value theorem.
  \end{frame}


\begin{frame}{Mean Value Theorem}
\begin{tikzpicture}
\begin{axis}[
  no markers, domain=0:10, samples=100,
  axis lines*=left, xlabel=$x$, ylabel=$y$,
  every axis y label/.style={at=(current axis.above origin),anchor=south},
  every axis x label/.style={at=(current axis.right of origin),anchor=west},
  height=5cm, width=12cm,
%  xtick={4,6.5}, ytick=\empty,
  xtick=\empty,
  ytick=\empty,
  enlargelimits=false, clip=false, axis on top,
  grid = major
  ]
  \foreach \x in {0,0.5,...,8}
  {
  \addplot [fill=cyan!0, draw=black, domain=\x:\x+1] {gauss(4,1)} \closedcycle;
  }
  \addplot [fill=cyan!20, draw=black, domain=3:3.5] {uniform(0.2419)} \closedcycle;
  
  \addplot [very thick,cyan!50!black] {gauss(4,1)};

  %  \addplot [very thick,cyan!50!black] {gauss(6.5,1)};
%\draw [yshift=-0.6cm, latex-latex](axis cs:4,0) -- node [fill=white] {$1.96\sigma$} (axis cs:5.96,0);
\end{axis}
  \draw (3.1,-0.3) node{$x_i$};
  \draw (3.8,-0.31) node{$x_{i+1}$};
  \draw (1.7,2.6) node{\color{cyan}$q_i>p_x(x_i)\delta$\cbla};
\end{tikzpicture}
\end{frame}


\begin{frame}{Mean Value Theorem}
\begin{tikzpicture}
\begin{axis}[
  no markers, domain=0:10, samples=100,
  axis lines*=left, xlabel=$x$, ylabel=$y$,
  every axis y label/.style={at=(current axis.above origin),anchor=south},
  every axis x label/.style={at=(current axis.right of origin),anchor=west},
  height=5cm, width=12cm,
%  xtick={4,6.5}, ytick=\empty,
  xtick=\empty,
  ytick=\empty,
  enlargelimits=false, clip=false, axis on top,
  grid = major
  ]
  \foreach \x in {0,0.5,...,8}
  {
  \addplot [fill=cyan!0, draw=black, domain=\x:\x+1] {gauss(4,1)} \closedcycle;
  }
  \addplot [fill=cyan!20, draw=black, domain=3:3.5] {uniform(0.3521)} \closedcycle;
  
  \addplot [very thick,cyan!50!black] {gauss(4,1)};

  %  \addplot [very thick,cyan!50!black] {gauss(6.5,1)};
%\draw [yshift=-0.6cm, latex-latex](axis cs:4,0) -- node [fill=white] {$1.96\sigma$} (axis cs:5.96,0);
\end{axis}
  \draw (3.1,-0.3) node{$x_i$};
  \draw (3.8,-0.31) node{$x_{i+1}$};
  \draw (1.7,2.6) node{\color{cyan}$q_i<p_x(x_{i+1})\delta$\cbla};
\end{tikzpicture}
\end{frame}


\begin{frame}{Mean Value Theorem}
\begin{tikzpicture}
\begin{axis}[
  no markers, domain=0:10, samples=100,
  axis lines*=left, xlabel=$x$, ylabel=$y$,
  every axis y label/.style={at=(current axis.above origin),anchor=south},
  every axis x label/.style={at=(current axis.right of origin),anchor=west},
  height=5cm, width=12cm,
%  xtick={4,6.5}, ytick=\empty,
  xtick=\empty,
  ytick=\empty,
  enlargelimits=false, clip=false, axis on top,
  grid = major
  ]
  \foreach \x in {0,0.5,...,8}
  {
  \addplot [fill=cyan!0, draw=black, domain=\x:\x+1] {gauss(4,1)} \closedcycle;
  }
  \addplot [fill=cyan!20, draw=black, domain=3:3.5] {uniform(0.3)} \closedcycle;
  
  \addplot [very thick,cyan!50!black] {gauss(4,1)};

  %  \addplot [very thick,cyan!50!black] {gauss(6.5,1)};
%\draw [yshift=-0.6cm, latex-latex](axis cs:4,0) -- node [fill=white] {$1.96\sigma$} (axis cs:5.96,0);
\end{axis}
  \draw (3.4,-0.3) node{$\bar{x}_i$};
  \draw (1.7,2.6) node{\color{cyan}$q_i=p_x(\bar{x}_i)\delta$\cbla};
\end{tikzpicture}
\end{frame}

\begin{frame}{Mean Value Theorem}
  The \textbf{Mean Value Theorem} tells us that assuming \crish$p_X(x)$\cbla{} is continuous we know we can
always pick a value of \crish$\bar{x}_i\in[x_i,x_{i+1}]$\cbla{} such that\crish
$$
  p(\bar{x}_i)\delta =q_i=\int_{x_i}^{x_{i+1}}p_x(x)dx
  $$
  \cbla
exactly.
\end{frame}

\begin{frame}{Integrals are gone}
  \crish
  $$H(X^\delta)=-\sum_i q_i \log_2{q_i} = -\sum_i  p_X(\bar{x}_i)\delta\log_2{p_X(\bar{x}_i)\delta}$$
    \cbla
    or, expanding out the log and using
    \crish
    $$
  \sum_ip_X(\bar{x}_i)\delta  = \int dx\, p_X(x)=1     
$$\cbla
    we get
  \crish
  $$H(X^\delta) = -\sum_i  p_X(\bar{x}_i)\delta\log_2{p_X(\bar{x}_i)}-\log_2{\delta}$$
    \cbla
\end{frame}

\begin{frame}{Use the Riemann approximation}
  \crish
$$
  \sum_i f(\bar{x}_i) \delta \rightarrow \int f(x)dx 
$$\cbla
as \crish $\delta \rightarrow 0$\cbla{}.
\end{frame}

\begin{frame}{Finally}
\crish
  $$H(X^\delta)+\log_2{\delta} = -\sum_i  p_X(\bar{x}_i)\delta\log_2{p_X(\bar{x}_i)}\rightarrow h(X)$$
    \cbla
    as \crish$\delta \rightarrow 0$\cbla.
\end{frame}
 
\begin{frame}{Shannon's entropy and differential entropy}
\crish
  $$H(X^\delta)+\log_2{\delta} \approx h(X)$$
    \cbla
    for small enough \crish$\delta $\cbla.
\end{frame}


\begin{frame}{Shannon's entropy and differential entropy}
\crish
  $$H(X^\delta)+\cblu\log_2{\delta}\crish{} \approx h(X)$$
    \cbla
    for small enough \crish$\delta $\cbla.
\end{frame}

\end{document}

