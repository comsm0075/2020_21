
\documentclass{beamer}
\usepackage[latin1]{inputenc}
%\usetheme{Montpellier}
%\usetheme{Boadilla}
%\usecolortheme[RGB={204,51,255}]{structure}
%\usecolortheme[named=purple]{structure}
\usecolortheme[RGB={62,128,62}]{structure}
%\definecolor{dark}{rgb}{0.3,0.15,0.3}
%\definecolor{light}{rgb}{0.8,0.6,0.8}
%\definecolor{reddish}{rgb}{.5,0.15,0.15}
\definecolor{dark}{rgb}{0.5,0.3,0.4}
%\definecolor{light}{rgb}{0.8,0.6,0.8}
\definecolor{reddish}{rgb}{.7,0.25,0.25}
\definecolor{greenish}{rgb}{.25,0.7,0.25}
\definecolor{blueish}{rgb}{.25,0.25,0.7}
\definecolor{purple}{rgb}{.5,0.0,0.5}
\usepackage{graphicx}
\usepackage{pstricks}

\setbeamertemplate{navigation symbols}{}

\newcommand{\crish}{\color{reddish}}
\newcommand{\cbla}{\color{black}}
\newcommand{\cred}{\color{red}}
\newcommand{\cblu}{\color{blue}}
\newcommand{\cgre}{\color{green}}

\newcommand{\sm}{\color{reddish}$}
\newcommand{\fm}{$\color{black}{}}

\newcommand{\letter}[1]{\color{blue}\texttt{#1}\color{black}}
\newcommand{\binary}[1]{\color{red}\texttt{#1}\color{black}}

\usepackage{tikz}
\usetikzlibrary{arrows,decorations.markings,positioning}
\usepackage{epstopdf}
\usetikzlibrary{fit}

\title[Information Theory lecture 5]{The data processing inequality: information theory lecture 5}
\author{COMSM0075 Information Processing and Brain}
\institute{\texttt{comsm0075.github.io}}
\date{September 2020}

\begin{document}

\maketitle

\begin{frame}{Snakes and Ladders or Moksha Patam}
  \begin{center}
    \includegraphics[width=8cm]{game.png}
  \end{center}
    \vfill
\tiny{\flushright{Image from wikipedia.}}
\end{frame}


\begin{frame}{start at 1}
  \begin{center}
    \includegraphics[width=8cm]{game1.jpg}
  \end{center}
    \vfill
\tiny{\flushright{Image from wikipedia.}}
\end{frame}


\begin{frame}{1$\rightarrow$5 (roll 4)}
  \begin{center}
    \includegraphics[width=8cm]{game5.jpg}
  \end{center}
    \vfill
\tiny{\flushright{Image from wikipedia.}}
\end{frame}


\begin{frame}{1$\rightarrow$15 (up the ladder)}
  \begin{center}
    \includegraphics[width=8cm]{game15.jpg}
  \end{center}
    \vfill
\tiny{\flushright{Image from wikipedia.}}
\end{frame}


\begin{frame}{1$\rightarrow$15$\rightarrow$19 (roll another 4)}
  \begin{center}
    \includegraphics[width=8cm]{game19.jpg}
  \end{center}
    \vfill
\tiny{\flushright{Image from wikipedia.}}
\end{frame}


\begin{frame}{1$\rightarrow$15$\rightarrow$19$\rightarrow$60 (up the ladder)}
  \begin{center}
    \includegraphics[width=8cm]{game60.jpg}
  \end{center}
    \vfill
\tiny{\flushright{Image from wikipedia.}}
\end{frame}


\begin{frame}{1$\rightarrow$15$\rightarrow$19$\rightarrow$60$\rightarrow$63 (roll a 3)}
  \begin{center}
    \includegraphics[width=8cm]{game63.jpg}
  \end{center}
    \vfill
\tiny{\flushright{Image from wikipedia.}}
\end{frame}


\begin{frame}{1$\rightarrow$15$\rightarrow$19$\rightarrow$60$\rightarrow$63$\rightarrow$99 (up the ladder)}
  \begin{center}
    \includegraphics[width=8cm]{game99.jpg}
  \end{center}
    \vfill
\tiny{\flushright{Image from wikipedia.}}
\end{frame}


\begin{frame}{(4,4,3)}
  \begin{center}
    \includegraphics[width=8cm]{game99.jpg}
  \end{center}
    \vfill
\tiny{\flushright{Image from wikipedia.}}
\end{frame}


\begin{frame}{()}
  \begin{center}
    \includegraphics[width=8cm]{game1.jpg}
  \end{center}
    \vfill
\tiny{\flushright{Image from wikipedia.}}
\end{frame}


\begin{frame}{(8)}
  \begin{center}
    \includegraphics[width=8cm]{game9.jpg}
  \end{center}
    \vfill
\tiny{\flushright{Image from wikipedia.}}
\end{frame}


\begin{frame}{(8,10)}
  \begin{center}
    \includegraphics[width=8cm]{game19.jpg}
  \end{center}
    \vfill
\tiny{\flushright{Image from wikipedia.}}
\end{frame}


\begin{frame}{(8,10)}
  \begin{center}
    \includegraphics[width=8cm]{game60.jpg}
  \end{center}
    \vfill
\tiny{\flushright{Image from wikipedia.}}
\end{frame}


\begin{frame}{(8,10,3)}
  \begin{center}
    \includegraphics[width=8cm]{game63.jpg}
  \end{center}
    \vfill
\tiny{\flushright{Image from wikipedia.}}
\end{frame}


\begin{frame}{(8,10,3)}
  \begin{center}
    \includegraphics[width=8cm]{game99.jpg}
  \end{center}
    \vfill
\tiny{\flushright{Image from wikipedia.}}
\end{frame}

\begin{frame}{Probability}

    \crish $X$\cbla, end of the first move, \crish $Y$\cbla, end of
    the second move and \crish$Z$\cbla{}, the end of the third after
    doing all the snakes and ladders stuff.
\vskip 1cm

  \begin{center}
    \includegraphics[width=3cm]{game.png}
  \end{center}
\end{frame}

\begin{frame}{Probability}
  \crish $$ p(Z=99)=0.0031 $$ \cbla
\end{frame}

\begin{frame}{Conditional probability - rolling a four}
    \crish $$ p(Z=99|X=15)=0.0046 $$ \cbla
  \begin{center}
    \includegraphics[width=7cm]{game15.jpg}
  \end{center}
\end{frame}


\begin{frame}{Conditional probability - rolling an eight}
    \crish $$ p(Z=99|X=9)=0.0046 $$ \cbla
  \begin{center}
    \includegraphics[width=7cm]{game9.jpg}
  \end{center}
\end{frame}


\begin{frame}{Conditional probability - rolling a 12}
    \crish $$ p(Z=99|X=13)=0.0077 $$ \cbla
  \begin{center}
    \includegraphics[width=7cm]{game13.jpg}
  \end{center}
\end{frame}

\begin{frame}{Conditional probability - rolling a seven}
    \crish $$ p(Z=99|X=34)=0 $$ \cbla
  \begin{center}
    \includegraphics[width=7cm]{game34.jpg}
  \end{center}
\end{frame}


\begin{frame}{Conditional probability - getting to 60}
    \crish $$ p(Z=99|Y=60)=0.0556 $$ \cbla
  \begin{center}
    \includegraphics[width=7cm]{game60.jpg}
  \end{center}
\end{frame}

\begin{frame}{Conditional probability - \crish$X$\cbla{} no longer matters}
    \crish $$ p(Z=99|Y=60)=p(Z=99|Y=60,X=15) $$ \cbla
\end{frame}


\begin{frame}{Conditional probability - \crish$X$\cbla{} no longer matters}
    \crish $$ p(Z=99|Y=60)=p(Z=99|Y=60,X=10) $$ \cbla
\end{frame}


\begin{frame}{Conditional independence}
\crish$X$\cbla{} and \crish$Z$\cbla{} are conditionally independent:
\crish
$$
p_{X,Z|Y}(x,z|y)=p_{X|Y}(x|y)p_{X|Z}(z|y)
$$
\cbla
\end{frame}


\begin{frame}{Markov chain}
  We write
 \crish
 $$X\rightarrow Y\rightarrow Z$$
 \cbla
 if \crish$X$\cbla{} and \crish$Z$\cbla{} are conditional indepedent, conditioned on \crish$Y$\cbla{}:
 \crish
$$
p_{X,Z|Y}(x,z|y)=p_{X|Y}(x|y)p_{X|Z}(z|y)
$$
\cbla
\end{frame}


\begin{frame}{Data processing inequality}
 if
 \crish
 $$X\rightarrow Y\rightarrow Z$$
 \cbla
then
 \crish
$$
I(X,Y)\ge I(X,Z)
$$
\cbla
with equality if and only if  \crish$X\rightarrow Z\rightarrow Y$\cbla
\end{frame}

\begin{frame}{Data processing inequality}
  \begin{quote}
    Processing extracts information, it doesn't add to it.
  \end{quote}
  \end{frame}



\end{document}

