\documentclass[12pt]{article}
\usepackage{amsfonts, epsfig}
\usepackage[authoryear]{natbib}
\usepackage{graphicx}
\usepackage{fancyhdr}
\pagestyle{fancy}
\lfoot{\texttt{comsm0075.github.io}}
\lhead{IP\&B - 1\_information\_theory - Conor}
\rhead{\thepage}
\cfoot{}
\begin{document}

\section*{Week one task} 

Alan Baddeley was a pioneering psychologist who help discover the
nature of working memory. One thing he did was examine the load
various tasks placed on working memory; in fact, because of this
expertise, he was consulted when the postcode system was designed, the
particular mixture of letters and numbers is particularly easy to
remember.

In
\begin{center}
\texttt{www.lancaster.ac.uk/staff/towse/rgpage.html}\end{center} his use of
random digit generation to assess cognitive load is described. It is
hard to produce random sequences of digits, so quantifying how random
a sequence is quantifies how hard a working memory task is.

The idea of this task is to try a simple version of this. Ask the
student to write down a sequence of digits at a rate of one digit a
second for two minutes, you will need to use a stop watch and say `now
now now . . .' at that rate. Next, give them a random five digit
number to remember and try the same thing again. Now, ask them to work
out the frequency of each digit in the two streams and hence an
estimate of $H(X)$. Are they different?

It would be interesting to collect all the answers you get from the
participants and then do a Wilcoxon signed-rank test to see if they
are different.

\end{document}

